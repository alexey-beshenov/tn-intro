\ifdefined\handout
  \documentclass[handout]{beamer}
\else
  \documentclass{beamer}
\fi

\usetheme{boxes}
\usecolortheme{structure}

\setbeamertemplate{footline}[frame number]

\ifdefined\handout
\definecolor{beamer@structure@color}{rgb}{0,0,0}
\setbeamertemplate{navigation symbols}{}
\setbeamercolor{normal text}{fg=black,bg=white}
\setbeamertemplate{frametitle}{\vskip 15pt\color{black}
\def\myhrulefill{\leavevmode\leaders\hrule height 1pt\hfill\kern 0pt}
\headingfont\insertframetitle\par\vskip-8pt\myhrulefill}
\else
\definecolor{beamer@structure@color}{rgb}{1,1,1}
\setbeamertemplate{navigation symbols}{}
\setbeamercolor{normal text}{fg=white,bg=black}
\setbeamertemplate{frametitle}{\vskip 15pt\color{white}
\def\myhrulefill{\leavevmode\leaders\hrule height 1pt\hfill\kern 0pt}
\headingfont\insertframetitle\par\vskip-8pt\myhrulefill}
\fi

\usepackage{amsmath,amssymb}

\newcommand{\NN}{\mathbb{N}}
\newcommand{\ZZ}{\mathbb{Z}}

\DeclareMathOperator{\mcd}{mcd}
\DeclareMathOperator{\mcm}{mcm}

\usepackage[spanish]{babel}

\usepackage{tikz-cd}
\usetikzlibrary{babel}
\usetikzlibrary{calc}

\usepackage{framed}

\newcommand{\dfn}{\mathrel{\mathop:}=}
\newcommand{\rdfn}{=\mathrel{\mathop:}}

\usepackage{mathspec}
\setsansfont[BoldFont={IBM Plex Sans Bold}, ItalicFont={IBM Plex Sans Italic}]{IBM Plex Sans}
\setmonofont[BoldFont={IBM Plex Mono Bold}, ItalicFont={IBM Plex Mono Italic}]{IBM Plex Mono}
\setmathrm[BoldFont={IBM Plex Sans Bold}, ItalicFont={IBM Plex Sans Italic}]{IBM Plex Sans}
\newfontfamily\headingfont[]{IBM Plex Sans Bold}

\setbeamercovered{transparent=10}


\begin{document}

\begin{frame}[plain,noframenumbering]
  \textbf{INTRODUCCIÓN A LA TEORÍA DE NÚMEROS}

  Alexey Beshenov $\mid$ \texttt{cadadr.org}

  \vfill

  \begin{center}\headingfont{\huge
      NUMERACIÓN POSICIONAL

      (EN LA BASE b)

      \vspace{1em}}

    \large (decimal, binaria, octal, hexadecimal, etc.)
  \end{center}

  \vfill
\end{frame}

\begin{frame}
  \frametitle{TEOREMA}

  \begin{framed}
    Fijemos un entero $b \ge 2$.

    \onslide<2->{Todo entero $a \ge 0$ puede ser escrito de modo único como
      \[ a = a_k\,b^k + a_{k-1}\,b^{k-1} + \cdots + a_2\,b^2 + a_1\,b + a_0 \]
      donde $0 \le a_i \le b-1$ y $a_k \ne 0$.}
  \end{framed}

  \onslide<3->{\textbf{Notación}:
    \[ a = \underline{a_k a_{k-1} \cdots a_2 a_1 a_0}_b. \]}
\end{frame}

\begin{frame}
  \frametitle{EJEMPLO}

  \begin{align*}
    123 & \onslide<2->{= \underline{123}_{10}} && \onslide<2->{= 10^2 + 2\cdot 10 + 3,} \\
        & \onslide<3->{= \underline{1111011}_2} && \onslide<3->{= 2^6 + 2^5 + 2^4 + 2^3 + 2 + 1,} \\
        & \onslide<4->{= \underline{11120}_3} && \onslide<4->{= 3^4 + 3^3 + 3^2 + 2\cdot 3,} \\
        & \onslide<5->{= \underline{1323}_4} && \onslide<5->{= 4^3 + 3\cdot 4^2 + 2\cdot 4 + 3,} \\
        & \onslide<6->{= \underline{443}_5} && \onslide<6->{= 4\cdot 5^2 + 4\cdot 5 + 3,} \\
        & \onslide<7->{= \underline{323}_6} && \onslide<7->{= 3\cdot 6^2 + 2\cdot 6 + 3,} \\
        & \onslide<8->{= \underline{234}_7} && \onslide<8->{= 2\cdot 7^2 + 3\cdot 7 + 4,} \\
        & \onslide<9->{= \underline{173}_8} && \onslide<9->{= 8^2 + 7\cdot 8 + 3,} \\
        & \onslide<10->{\cdots}
  \end{align*}
\end{frame}

\begin{frame}
  \frametitle{DEMOSTRACIÓN: EXISTENCIA}

  \onslide<2->{División con residuo, hasta obtener el cociente $q_{k+1} = 0$:}
  \begin{align*}
    \onslide<3->{a} & \onslide<3->{= b q_0 + a_0} \\
                    & \onslide<4->{= b \, (b q_1 + a_1) + a_0} \\
                    & \onslide<5->{= b \, (b \, (b q_2 + a_2) + a_1) + a_0} \\
                    & \onslide<6->{= b \, (b \, (b \, (b q_3 + a_3) + a_2) + a_1) + a_0} \\
                    & \onslide<7->{= \cdots} \\
                    & \onslide<8->{= a_k\,b^k + a_{k-1}\,b^{k-1} + \cdots + a_2\,b^2 + a_1\,b + a_0. \qed}
  \end{align*}
\end{frame}

\begin{frame}
  \frametitle{EJEMPLO: 123 EN LA BASE 3}

  \begin{align*}
    123 & \onslide<2->{= 3\cdot 41 + \boxed{0}} \\
        & \onslide<3->{= 3\cdot (3\cdot 13 + \boxed{2}) + \boxed{0}} \\
        & \onslide<4->{= 3\cdot (3\cdot (3\cdot 4 + \boxed{1}) + \boxed{2}) + \boxed{0}} \\
        & \onslide<5->{= 3\cdot (3\cdot (3\cdot (3\cdot \boxed{1} + \boxed{1}) + \boxed{1}) + \boxed{2}) + \boxed{0}} \\
        & \onslide<6->{= 1\cdot 3^4 + 1\cdot 3^3 + 1\cdot 3^2 + 2\cdot 3 + 0} \\
        & \onslide<7->{= \underline{11120}_3.}
  \end{align*}
\end{frame}

\begin{frame}
  \frametitle{DEMOSTRACIÓN: UNICIDAD}

  \onslide<2->{\[ a_0 + a_1\,b + a_2\,b^2 + \cdots + a_k\,b^k = a_0' + a_1'\,b + a_2'\,b^2 + \cdots + a_k'\,b^k. \]}

  \onslide<3->{\[ b \mid a_0 - a_0', \quad 0 \le |a_0 - a_0'| \le b-1. \]
  Entonces, $a_0 = a_0'$.}

  \onslide<4->{\begin{align*}
    a_1\,b + a_2\,b^2 + \cdots + a_k\,b^k & = a_1'\,b + a_2'\,b^2 + \cdots + a_k'\,b^k, \\
    a_1 + a_2\,b + \cdots + a_k\,b^{k-1} & = a_1' + a_2'\,b + \cdots + a_k'\,b^{k-1}.
  \end{align*}}

  \onslide<5->{$a_1 = a_1'$, etc\dots \qed}
\end{frame}

\begin{frame}
  \frametitle{BASES COMUNES}

  \onslide<2->{En la vida cotidiana:
  \begin{itemize}
  \item \textbf{decimal}: $b = 10$.
  \end{itemize}}

  \onslide<3->{En las matemáticas babilónicas (ca. 2000 a.C.):
  \begin{itemize}
  \item \textbf{sexagesimal}: $b = 60$.
  \end{itemize}}

  \onslide<4->{En la informática:}
  \begin{itemize}
  \item<4-> \textbf{binaria}: $b = 2$,

  \item<5-> \textbf{octal}: $b = 8$,

  \item<6-> \textbf{hexadecimal}: $b = 16$.

    Los dígitos más allá de $9$ se denotan por

    \texttt{A} (10), \texttt{B} (11), \texttt{C} (12), \texttt{D} (13), \texttt{E} (14), \texttt{F} (15).

    \onslide<7->{\textbf{Ejemplo}:
    \[ 2022 = 7\cdot 16^2 + 14 \cdot 16 + 6
      = \underline{\texttt{7E6}}_{16}. \]}
  \end{itemize}
\end{frame}

\begin{frame}
  \frametitle{* NÚMEROS REALES}

  \onslide<2->{\[
    x = a_k a_{k-1} \cdots a_1 a_0, a_{-1} a_{-2} a_{-3} \cdots
    \longleftrightarrow
    x = \sum_i a_i\cdot b^i.
  \]}

  \begin{itemize}
  \item<3-> Los dígitos no son necesariamente únicos:
    $$1,000000\ldots = 0,999999\ldots$$

  \item<4-> $x$ es \textbf{racional} si y solo si los dígitos son
    \textbf{eventualmente periódicos}: existe $n$ tal que
    $a_{-i} = a_{-(i+n)}$ para todo $i$ suficientemente grande.

    \onslide<5->{\textbf{Ejemplo}:
    \begin{align*}
      \frac{1}{12} & = 0.0833333333\ldots \\
          \sqrt{2} & = 1.4142135623\ldots \\
          \text{π} & = 3.1415926535\ldots
    \end{align*}}
  \end{itemize}
\end{frame}

\begin{frame}
  \frametitle{¿CUÁNTOS DÍGITOS NECESITAMOS?}

  \begin{itemize}
  \item<2-> Si $a$ tiene $n$ dígitos $a_0,a_1,\ldots,a_{n-1}$
    en la base $b$:
    \[ a = a_{n-1}\,b^{n-1} + \cdots + a_1\,b + a_0,
      \quad
      a_{n-1} \ne 0. \]

  \item<3-> El número más grande de $n$ dígitos:
    \[ (b-1)\,b^{n-1} + \cdots + (b-1)\,b + (b-1) = (b-1)\,\frac{b^n - 1}{b-1} = b^n - 1. \]

  \item<4-> Cotas:
    \begin{gather*}
      b^{n-1} \le a < b^n, \\
      \onslide<5->{n-1 \le \log_b (a) < n.}
    \end{gather*}

  \item<6-> Conclusión:
    \[ n = \lfloor\log_b (a)\rfloor + 1. \]
  \end{itemize}
\end{frame}

\begin{frame}
  \frametitle{EJEMPLO}

  \begin{align*}
    \onslide<2->{10^1} & \onslide<2->{= \underline{\text{\texttt{1010}}}_2,} & \onslide<2->{\log_2 (10^1)} & \onslide<2->{\approx 3.32,} \\
    \onslide<3->{10^2} & \onslide<3->{= \underline{\text{\texttt{1100100}}}_2,} & \onslide<3->{\log_2 (10^2)} & \onslide<3->{\approx 6.64,} \\
    \onslide<4->{10^3} & \onslide<4->{= \underline{\text{\texttt{1111101000}}}_2,} & \onslide<4->{\log_2 (10^3)} & \onslide<4->{\approx 9.96,} \\
    \onslide<5->{10^4} & \onslide<5->{= \underline{\text{\texttt{10011100010000}}}_2,} & \onslide<5->{\log_2 (10^4)} & \onslide<5->{\approx 13.28,} \\
    \onslide<6->{10^5} & \onslide<6->{= \underline{\text{\texttt{11000011010100000}}}_2,} & \onslide<6->{\log_2 (10^3)} & \onslide<6->{\approx 16.60,} \\
    \onslide<7->{10^6} & \onslide<7->{= \underline{\text{\texttt{11110100001001000000}}}_2,} & \onslide<7->{\log_2 (10^6)} & \onslide<7->{\approx 19.93,} \\
       & \onslide<8->{\cdots}
  \end{align*}
\end{frame}

\begin{frame}
  \frametitle{LA BASE NO ES MUY IMPORTANTE}

  \begin{itemize}
  \item<2-> Si pasamos de la base $b$ a $b' > b$, el número de dígitos se
    disminuye proporcionalmente con factor $\sim \log_b (b')$.

  \item<3-> Usamos la base $10$ porque tenemos diez dedos.

  \item<4-> Si tuviéramos ocho dedos en cada mano, usaríamos la base
    hexadecimal. Esto nos daría una economía en dígitos de
    $\log_{10} (16) = 1.204119\ldots$ alrededor de $20\%$.

    \vspace{0.5em}

    \onslide<5->{\textbf{Ejemplo}:\small
      \begin{align*}
        20! & = 265252859812191058636308480000000 && (33\text{ dígitos}) \\
            & = \underline{\text{\texttt{D13F6370F96865DF5DD54000000}}}_{16} && (27\text{ dígitos})
      \end{align*}}

  \item<6-> La elección de base no es muy
    importante, lo importante es no usar la \textbf{base unaria}
    \[ 1 = |, ~ 2 = ||, ~ 3 = |||, ~ 4 = ||||, ~ \ldots \]
  \end{itemize}
\end{frame}

\begin{frame}
  \frametitle{CRITERIOS DE DIVISIBILIDAD}

  \onslide<2->{Consideremos $a = \underline{a_k a_{k-1} \cdots a_1 a_0}_{10}$.}

  \begin{itemize}
  \item<3-> $2\mid a$ si y solamente si el último dígito de $a$ es par
    ($0, 2, 4, 6, 8$).

  \item<4-> $5\mid a$ si y solamente si el último dígito $a_0$ es $0$ o $5$.

  \item<5-> $10\mid a$ si y solamente si el último dígito $a_0$ es $0$.
  \end{itemize}

  \onslide<6->{\textbf{Demostración}: pequeño ejercicio.}
\end{frame}

\begin{frame}
  \frametitle{CRITERIO DE DIVISIBILIDAD POR 3}

  \onslide<2->{\begin{framed}
    Para $a = \underline{a_k a_{k-1} \cdots a_1 a_0}_{10}$ tenemos
    $3 \mid a \iff 3 \mid \sum_i a_i$.
  \end{framed}}

  \onslide<3->{\textbf{Demostración} («reducción módulo 3»).}

  \begin{itemize}
  \item<4-> Para cualquier $i \ge 1$, la división de $10^i$ por $3$ da residuo $1$:
    \onslide<5->{\[
        \frac{10^i - 1}{3} = 3\,\frac{10^i - 1}{10 - 1}
        = 3\,(1 + 10 + 10^2 + \cdots + 10^{i-1}) \in \ZZ.
      \]}

  \item<6-> Podemos escribir
    \begin{align*}
      a & = 10^k\,a_k + \cdots + 10^2\cdot a_2 + 10\cdot a_1 + a_0 \\
        & \onslide<7->{= (3\cdot q_k + 1)\,a_k + \cdots + (3\cdot q_2 + 1)\,a_2 + (3\cdot q_1 + 1)\,a_1 + a_0} \\
        & \onslide<8->{= a_0 + a_1 + a_2 + \cdots + a_k + (\text{algo divisible por }3). \qed}
    \end{align*}
  \end{itemize}
\end{frame}

\begin{frame}[plain,noframenumbering]

  \vfill

  \begin{center}\huge\headingfont
    EJERCICIOS
  \end{center}

  \vfill
\end{frame}

\begin{frame}
  \frametitle{EJERCICIO 1: UN PAR DE CRITERIOS DE DIVISIBILIDAD}

  \onslide<2->{Consideremos
  \[ a = \underline{a_k a_{k-1} \cdots a_1 a_0}_{10}. \]}

  \begin{itemize}
  \item<3-> $4\mid a$ si y solamente si los últimos dos dígitos forman un número
    $a_1 a_0$ que es divisible por $4$.
    \onslide<4->{\textbf{Ejemplo}:
    \[ 14156 = 4\cdot 3539, \quad 4 \mid 56. \]}

  \item<5-> $11\mid a$ si y solamente si la suma alternante de los dígitos
    $\sum_i (-1)^i\,a_i$ es divisible por $11$.
    \onslide<6->{\textbf{Ejemplo}:
    \[ 87109 = 7919\cdot 11, \quad 8 - 7 + 1 - 0 + 9 = 11. \]}

  \item[*]<7-> Existen criterios de divisibilidad por $7$, $13$, etc.

    Son más complicados y al mismo tiempo bastante inútiles.
  \end{itemize}
\end{frame}

\begin{frame}
  \frametitle{EJERCICIO 2: SUMAS DE DIFERENTES POTENCIAS DE 3}

  \onslide<2->{Consideremos los números que son sumas de diferentes potencias de $3$:}
  \begin{align*}
    \onslide<3->{a_1} & \onslide<3->{= 3^0 = 1,} \\
    \onslide<4->{a_2} & \onslide<4->{= 3^1 = 3,} \\
    \onslide<5->{a_3} & \onslide<5->{= 3^0 + 3^1 = 4,} \\
    \onslide<6->{a_4} & \onslide<6->{= 3^2 = 9,} \\
    \onslide<7->{a_5} & \onslide<7->{= 3^0 + 3^2 = 10,} \\
    \onslide<8->{a_6} & \onslide<8->{= 3^1 + 3^2 = 12,} \\
    \onslide<9->{a_7} & \onslide<9->{= 3^0 + 3^1 + 3^2 = 13,} \\
                      & \onslide<10->{\cdots}
  \end{align*}
  \onslide<11->{Encuentre el término $a_{100}$ en esta sucesión.}
\end{frame}

\begin{frame}
  \frametitle{EJERCICIO 3: BASE 3 CON DÍGITOS -1, 0, +1}

  \onslide<2->{\begin{framed}
    Todo entero $a$ puede ser escrito de manera única como
    $$a = a_k\,3^k + \cdots + a_2\,3^2 + a_1\,3 + a_0,$$
    donde $a_i \in \{ -1, 0, +1 \}$ y $a_k \ne 0$.
  \end{framed}}

  \onslide<3->{Por ejemplo,
    \[
      17 = 3^3 - 3^2 - 1,
      \quad
      57 = 3^4 - 3^3 + 3.
    \]}

  \onslide<4->{¿Cómo se expresa $a = 123$ en esta base?}

  \onslide<5->{Demuestre lo mismo para
    $$a = a_k\,5^k + \cdots + a_2\,5^2 + a_1\,5 + a_0$$
    y $a_i \in \{ -2, -1, 0, +1, +2 \}$. Generalice este resultado.}
\end{frame}

\begin{frame}
  \frametitle{EJERCICIO 4: BASE FACTORIAL}

  \onslide<2->{\begin{framed}
    Todo entero $a \ge 0$ puede ser escrito de manera única como
    $$a = a_k\,k! + a_{k-1}\,(k-1)! + \cdots + a_2\,2! + a_1,$$
    donde $0 \le a_i \le i$ y $a_k \ne 0$.
  \end{framed}}

  \onslide<3->{Aquí el \textbf{factorial} está definido por
  $n! \dfn 1\cdot 2\cdots (n-1)\cdot n$.}

  \onslide<4->{Por ejemplo,
    \[
      17 = 2\cdot 3! + 2\cdot 2! + 1,
      \quad
      57 = 2\cdot 4! + 3! + 2! + 1.
    \]}

  \onslide<5->{¿Cómo se expresa $a = 123$ en esta base?}
\end{frame}

\begin{frame}
  \frametitle{EJERCICIO 5: FRACCIÓN CON CEROS Y UNOS}

  \onslide<2->{La fracción
  $$\frac{101010101}{110010011}$$
  tiene el mismo valor si «$1$» en el medio del númerador y denominador se
  remplaza por un número impar de $1$'s.}

  \[
    \onslide<3->{\frac{101010101}{110010011} =
      \frac{10101110101}{11001110011}}
    \onslide<4->{= \frac{1010111110101}{1100111110011}}
    \onslide<5->{= \cdots}
  \]

  \onslide<6->{Esto es válido en cualquier base $b$.}
\end{frame}

\begin{frame}[plain,noframenumbering]

  \vfill

  \begin{center}\huge\headingfont
    ¡GRACIAS POR SU ATENCIÓN!
  \end{center}

  \vfill
\end{frame}
\end{document}
