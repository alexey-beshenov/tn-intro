\ifdefined\handout
  \documentclass[handout]{beamer}
\else
  \documentclass{beamer}
\fi

\usetheme{boxes}
\usecolortheme{structure}

\setbeamertemplate{footline}[frame number]

\ifdefined\handout
\definecolor{beamer@structure@color}{rgb}{0,0,0}
\setbeamertemplate{navigation symbols}{}
\setbeamercolor{normal text}{fg=black,bg=white}
\setbeamertemplate{frametitle}{\vskip 15pt\color{black}
\def\myhrulefill{\leavevmode\leaders\hrule height 1pt\hfill\kern 0pt}
\headingfont\insertframetitle\par\vskip-8pt\myhrulefill}
\else
\definecolor{beamer@structure@color}{rgb}{1,1,1}
\setbeamertemplate{navigation symbols}{}
\setbeamercolor{normal text}{fg=white,bg=black}
\setbeamertemplate{frametitle}{\vskip 15pt\color{white}
\def\myhrulefill{\leavevmode\leaders\hrule height 1pt\hfill\kern 0pt}
\headingfont\insertframetitle\par\vskip-8pt\myhrulefill}
\fi

\usepackage{amsmath,amssymb}

\newcommand{\NN}{\mathbb{N}}
\newcommand{\ZZ}{\mathbb{Z}}

\DeclareMathOperator{\mcd}{mcd}
\DeclareMathOperator{\mcm}{mcm}

\usepackage[spanish]{babel}

\usepackage{tikz-cd}
\usetikzlibrary{babel}
\usetikzlibrary{calc}

\usepackage{framed}

\newcommand{\dfn}{\mathrel{\mathop:}=}
\newcommand{\rdfn}{=\mathrel{\mathop:}}

\usepackage{mathspec}
\setsansfont[BoldFont={IBM Plex Sans Bold}, ItalicFont={IBM Plex Sans Italic}]{IBM Plex Sans}
\setmonofont[BoldFont={IBM Plex Mono Bold}, ItalicFont={IBM Plex Mono Italic}]{IBM Plex Mono}
\setmathrm[BoldFont={IBM Plex Sans Bold}, ItalicFont={IBM Plex Sans Italic}]{IBM Plex Sans}
\newfontfamily\headingfont[]{IBM Plex Sans Bold}

\setbeamercovered{transparent=10}
